\documentclass[12pt]{article}
%% arXiv paper template by Flip Tanedo
%% last updated: Dec 2016



%%%%%%%%%%%%%%%%%%%%%%%%%%%%%
%%%  THE USUAL PACKAGES  %%%%
%%%%%%%%%%%%%%%%%%%%%%%%%%%%%

\usepackage{amsmath}
\usepackage{amssymb}
\usepackage{amsfonts}
\usepackage{graphicx}
\usepackage{xcolor}
\usepackage{nopageno}
\usepackage{enumerate}
\usepackage{parskip}


\renewcommand{\thesection}{}
\renewcommand{\thesubsection}{\arabic{subsection}}

%%%%%%%%%%%%%%%%%%%%%%%%%%%%%%%%%%%%%%%%%%%%%%%
%%%  PAGE FORMATTING and (RE)NEW COMMANDS  %%%%
%%%%%%%%%%%%%%%%%%%%%%%%%%%%%%%%%%%%%%%%%%%%%%%

\usepackage[margin=2cm]{geometry}   % reasonable margins

\graphicspath{{figures/}}	        % set directory for figures

% for capitalized things
\newcommand{\acro}[1]{\textsc{\MakeLowercase{#1}}}    

\numberwithin{equation}{section}    % set equation numbering
\renewcommand{\tilde}{\widetilde}   % tilde over characters
\renewcommand{\vec}[1]{\mathbf{#1}} % vectors are boldface

\newcommand{\dbar}{d\mkern-6mu\mathchar'26}    % for d/2pi
\newcommand{\ket}[1]{\left|#1\right\rangle}    % <#1|
\newcommand{\bra}[1]{\left\langle#1\right|}    % |#1>
\newcommand{\Xmark}{\text{\sffamily X}}        % cross out

\let\olditemize\itemize
\renewcommand{\itemize}{
  \olditemize
  \setlength{\itemsep}{1pt}
  \setlength{\parskip}{0pt}
  \setlength{\parsep}{0pt}
}


% Commands for temporary comments
\newcommand{\comment}[2]{\textcolor{red}{[\textbf{#1} #2]}}
\newcommand{\flip}[1]{{\color{red} [\textbf{Flip}: {#1}]}}
\newcommand{\email}[1]{\texttt{\href{mailto:#1}{#1}}}

\newenvironment{institutions}[1][2em]{\begin{list}{}{\setlength\leftmargin{#1}\setlength\rightmargin{#1}}\item[]}{\end{list}}


\usepackage{fancyhdr}		% to put preprint number



% Commands for listings package
%\usepackage{listings}      % \begin{lstlisting}, for code
%
% \lstset{basicstyle=\ttfamily\footnotesize,breaklines=true}
%    sets style to small true-type



%%%%%%%%%%%%%%%%%%%
%%%  HYPERREF  %%%%
%%%%%%%%%%%%%%%%%%%

%% This package has to be at the end; can lead to conflicts
\usepackage{microtype}
\usepackage[
	colorlinks=true,
	citecolor=black,
	linkcolor=black,
	urlcolor=green!50!black,
	hypertexnames=false]{hyperref}





\begin{document}


\begin{center}

    {\Large \textsc{Short HW 1}:
    \textbf{Jupyter}}
    
\end{center}

\vskip .4cm

\noindent
\begin{tabular*}{\textwidth}{rl}
	\textsc{Course:}& Physics 177, \emph{Computational Physics} (2018)
	\\
	\textsc{Instructor:}& Prof. Flip Tanedo (\email{flip.tanedo@ucr.edu})
	\\
	\textsc{Due by:}& \textbf{Thursday}, April 5
\end{tabular*}

\noindent
Note that this short assignment is due in class on Thursday. You have only \emph{two days} to do it. This should be quick, I recommend doing it right after class on Tuesday.

\subsection{Install Jupyter}

This class will use \textbf{Python 3} in the \textbf{Jupyter} environment. By \emph{Thursday}, please make sure you have this set up on your machine. 

\noindent\textsc{Note}: Prof.~Tanedo is \emph{not} available for technical support for this. 

\subsubsection{Install Python}
If you do not already have Python 3 installed, please install it. Here is one such tutorial:
\begin{itemize}
	\item \textbf{Mac}: \url{http://docs.python-guide.org/en/latest/starting/install3/osx/}
	\item \textbf{Win}: \url{http://docs.python-guide.org/en/latest/starting/install3/win/}
	\item \textbf{Linux}: \url{http://docs.python-guide.org/en/latest/starting/install3/linux/}
\end{itemize}
Another easy way to install Python is to use Anaconda: \url{https://www.anaconda.com/download/} \ .


At the command line, type in \texttt{python --version} or (if you have multiple versions installed) \texttt{python3 --version}. Write what the output is.

\noindent\textsc{Hint}: If the output looks like \texttt{Python  2.7.10}, then you \emph{do not} have the correct version of Python.
\vspace{2em}

\subsubsection{Install Jupyter}

\textbf{Jupyter} is a `notebook' front-end that makes it easy to write, test, and share code. Please install it using the instructions here: \url{http://jupyter.org/install}

Once installed, run the Jupyter notebook environment by entering \texttt{jupyter notebook} a the command line. This will open a web browser. Write the url that pops up. \textsc{Hint}: it should start with \texttt{localhost}.

\end{document}